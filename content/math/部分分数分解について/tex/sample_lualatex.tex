% sample_lualatex.tex
% LuaLaTeX 用のサンプルファイル
% コンパイル例: lualatex sample_lualatex.tex

\documentclass[a4paper,11pt]{ltjsarticle}

% 日本語フォント(LuaLaTeX)
\usepackage{luatexja-fontspec}

% 日本語フォント設定(macOS想定)
% IPAex を使う場合(TeX Liveに含まれている)
\setmainjfont{IPAexMincho}[
  BoldFont=IPAexGothic,
  ItalicFont=IPAexMincho,
  BoldItalicFont=IPAexGothic
]

% 数式
\usepackage{amsmath,amssymb,amsthm,bm}
\newtheorem{theorem}{定理}

\title{LuaLaTeX サンプル:部分分数分解について}
\author{真中 遥道}
\date{2021年4月30日}

\begin{document}

\maketitle

これは LuaLaTeX でコンパイルするためのサンプルです。
日本語と数式が PDF.js 上でも正しく表示されるかを確認するために作成しました。

\section{部分分数分解}

塾講師のアルバイトで有理関数の積分を教えるときに、
部分分数分解についてある知見を得た。
調べてみたところ割と有名事実みたいだが、
自力で証明できたのが嬉しかったので記録を残しておく。

複素数体 $\mathbb{C}$ 上の多項式 $f(x), g(x)$ を考える。
$\deg f = m$, $\deg g = n$ とし、$g(x)$ の因数分解を
\[
  g(x) = a \prod_{i=1}^{k} (x - \alpha_i)^{\lambda_i}
  \quad (a \neq 0,\; i \neq j \Rightarrow \alpha_i \neq \alpha_j)
\]
と書けるとする。

\begin{theorem}[部分分数分解の定理]
  任意の有理式 $f(x)/g(x)$ に対して複素数
  $a_0, \ldots, a_{m-n}, b_{1,1}, \ldots, b_{k,\lambda_k}$ が存在し,
  \[
    \frac{f(x)}{g(x)}
    = \sum_{i=0}^{m-n} a_i x^i
    + \sum_{i=1}^{k} \sum_{j=1}^{\lambda_i}
    \frac{b_{i,j}}{(x - \alpha_i)^j}
  \]
  と部分分数分解できる。
  なお $m - n < 0$ のとき右辺第一項は空和として $0$ とみなす。
\end{theorem}

この PDF を `pdffonts` コマンドで確認すると,
LuaLaTeX によって埋め込まれたフォントに対して
`uni = yes` となっていることが期待される。
その状態であれば,PDF.js 経由でブラウザに埋め込んだときも
日本語と数式が正しく表示されやすくなる。

\end{document}
